\documentclass[11pt,a4j]{jarticle}

\input{include/macro.tex}
\input{include/preamble.tex}

\begin{document}
\begin{titlepage}

  \vspace*{25mm}

  \begin{center}
    {\huge 知能制御PBL\\}
    \vspace{10mm}
    {\Huge 第3回RCR中間報告書\\}
    \vspace{20mm}
    {\Large 2018年6月27日}

    \vspace{15mm}

    {\LARGE 西田研究室\\}

    \vspace{15mm}

    {\Large
   13104042 烏谷崇大 \\
   14104055 佐々木秀将\\
   15104021 長田駿二郎\\
   15104026 川崎雄太朗\\
   15104050 坂元勇太 \\
   15104081 徳野将士 \\
   15104113 前田修一 \\
   15104117 右田明花 \\
   15104134 山福佳  \\
   17104311 横田篤紀 \\
}

  \end{center}

\end{titlepage}


\newpage
	\tableofcontents

\newpage
\section{目的}
	学部3年までに学習した制御理論や電気回路・情報工学の知識を使って,競技場内を自律的に走行するロボットの製作を行う.
	研究室で一丸となってプロジェクトを進行し,共同で課題を達成することの難しさや楽しさを学び,エンジニアとして仕事を進めるための素養を身に付ける.

\section{RCR (Robot Car Race) 2018}

\subsection{競技概要}
	ウレタンパネルを用いてレイアウトされる周回コースにおいて,格子模様のコントロールライン手前から走行開始してコントロールラインを3回通過後に停止するまでの時間を競うロボットカー(ロボカー)を製作する.

\subsection{コース}
	一辺$50\unit{cm}$の正方形及び扇形の黒色ウレタンパネルと白黒格子模様のコントロールライン付ウレタンパネルを組み合わせ,コースを構成する.
	なお,競技会当日までコースは公表されない.また,コースフェンスの高さが低いため,コースフェンスのコース側に高さ$10\unit{cm}$の壁を設置する.

\subsection{競技ルール}
	\begin{enumerate}
      \item コントロールライン手前からの走行開始からコントロールラインを3回通過後に停止するまでの時間を競う.
      \item 各チームあたり10分以内に最大3回走行し,最短時間の走行を評価する.
      \item コース2周以上走行すること.
      \item 競技会当日のコース試走は認めない.
      \item ロボカーがコース周囲の壁に接触した場合は失格とする.
      \item コース及び壁に物を設置したり,手を加えてはいけない.
      \item コース内に足を踏み入れないこと.
    \end{enumerate}

\newpage

\section{ROS (Robot Operating System)}
\subsection{ROSとは}
ROS(Robot Operating System)とはOpen Source Robotics Foundationによって管理されているソフトウェア開発者のロボット・アプリケーション作成を支援するフレームワークである.
具体的には,ハードウェア抽象化,デバイスドライバ,ライブラリ,視覚化ツール,メッセージ通信,パッケージ管理などが提供されている.つまりROSは汎用コンピュータ向けのOSではなく,汎用コンピュータ向けOS上で動作するメタOSとして捉えることができる\cite{kurazume}.

\refig{ros_topic}に示すようにROSではプロセス(実行プログラム)はノードという単位で扱い,ノード間の通信はトピックと呼ばれる``Publisher/Subscriber''モデルで実現される\cite{ogura}.

これにより,プログラミング言語や通信相手さえ意識することなく簡単にプロセス間通信を実現できる.
これは各ノード間のインタフェース,すなわちトピックの名前と型さえ決定すればノードごとに独立して開発を行うことができるという利点でもある.\\

以上の利点を考慮し,本研究室ではROSがインストール可能なマイコンボードであるRaspberryPi3 Model B上にROSをインストールして開発を進めていくこととした.

\begin{figure}[htb]
  \centering
    \includegraphics[width=0.5\hsize]{picture/eps/ros_topic.eps}
    \caption{ROSノードとトピックの概念}
    \label{fig::ros_topic}
\end{figure}



\begin{figure}[htb]
  \centering
    \includegraphics[width=0.8\hsize]{picture/eps/ros_nodes.eps}
    \caption{ROSノードとトピックの構成}
    \label{fig::ros_nodes}
\end{figure}

\newpage
\subsection{ROSノードとトピックの構成}
\refig{ros_nodes}に開発するROSノードとトピックの構成を示す.各ノードの役割は次の通りである.
\begin{description}

    \item[距離センサデータ取得] \mbox{} \\
      ロボカーの前方及び両側面に設置した距離センサからシリアルバス規格の一つである$\mathrm{I^2C}$を介して距離データを$\mathrm{[mm]}$単位で取得し外れ値処理や正規化を施した後にPublishする.
    \item[コントロールライン検出] \mbox{} \\
      ロボカーの後方下部に設置したフォトリフレクタによってコントロールラインを通過した回数をカウントしPublishする.

    \item[タイヤ角速度データ取得] \mbox{} \\
      ロボカーの後方に設置したロータリーエンコーダによって計測したタイヤの回転角を基に,タイヤの回転角速度を算出してPublishする.

    \item[DCモータ目標値生成] \mbox{} \\
      ロボカーの前方方向の距離データをSubscribeし,それをもとにDCモータに与える目標値を生成しPublishする.

    \item[ドライバ] \mbox{} \\
      DCモータに与える目標値,ロボカーの両側面の壁との距離,タイヤの角速度,周回数をSubscribeし,サーボモータの目標値を生成し,サーボモータを駆動させる.また,DCモータを目標値に追従するようなPI制御系によって駆動する.さらに,規定の周回数になるとロボカーを停止させる.


  \end{description}



\begin{thebibliography}{9}
 \bibitem{kurazume}
    表允晳,倉爪亮,渡邊裕太, ``詳説 ROSロボットプログラミング-導入からSLAM・Gazebo・MoveItまで-'', \\
    Kurazume Laboratory, pp.15-18, (2015).

  \bibitem{ogura}
    小倉崇, ``ROSではじめるロボットプログラミング'', 工学社, pp.8-10, (2015).

\end{thebibliography}

\end{document}
