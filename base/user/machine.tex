\section{機体}
\refig{first}に配布されたロボカー本体の写真をしめす.\refig{fisrt}より本体の前方には前輪を駆動させ機体を旋回させるためのサーボモータを設置している.そして,本体の後方には後輪を駆動させ機体を動かすためのDCモータを設置し,その上にロータリエンコーダを設置した.

次に,\refig{second}に本体の上に設置するパーツの写真を,\refig{third}にその上に設置するパーツの写真示す.\refig{second}より中央にはDCモータを制御するためのモータドライバ,後方には各部品に電力を供給するためのバッテリ,前方にはRaspberry Pi3 model Bに電力を供給するためのモバイルバッテリを設置した.そして,前方と左右にコースの壁の距離を計測するためのPSDセンサを設置した.\refig{third}より,一番上にはRaspberry Pi3 model BとDC-DCコンバータを含めた電気回路を設置している.

そして,現在,配線が\refig{second},\refig{third}よりロボカー本体に他の部品を設置するためにユニバーサルプレートを採用した.その理由は最初に,加工しやすいからである.ユニバーサルプレートは素材がプラスチックで出来ているので切断しやすいという特徴がある.次に,様々な部品の取り外しや設置がやりやすいからである.ユニバーサルプレートには無数の穴が開いているので簡単にボルトとナットに部品を設置できる.また,プラスチックは電気を通しにくいので電気回路の絶縁にもなるからである.

最後に,\refig{forth}にロボカーの背面の写真を示す.\refig{forth}より,現在はまだ設置してないがロボカーの背面にゴールラインを読み取るためのフォトリフレクタを設置する予定である.

\begin{figure}[htb]
\centering
\includegraphics[width=0.5\hsize]{picture/eps/one.eps}
\caption{ロボカー本体}
\label{fig::first}
\end{figure}

\begin{figure}[htb]
\centering
\includegraphics[width=0.5\hsize]{picture/eps/two.eps}
\caption{ロボカー1段目}
\label{fig::second}
\end{figure}

\begin{figure}[htb]
\centering
\includegraphics[width=0.5\hsize]{picture/eps/three.eps}
\caption{ロボカー2段目}
\label{fig::third}
\end{figure}

\begin{figure}[htb]
\centering
\includegraphics[width=0.5\hsize]{picture/eps/four.eps}
\caption{ロボカー背面}
\label{fig::forth}
\end{figure}
