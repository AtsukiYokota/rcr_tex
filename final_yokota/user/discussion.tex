\section{考察}
ここでは1回目の走行において壁に衝突した原因,3回目の走行においてコントロールラインを一回検出し損ねた原因,またステアリング制御が振動を起こした原因の3点について考察していく.

\subsection{壁に衝突した原因}
1回目の走行では,2周目の終盤までは安定して走行していたが,終盤のU字カーブでステアリングを切ることなく壁に衝突した.この時,一切ステアリング動作をしていなかったため,距離センサの異常により距離データの取得に失敗していたと考えられる.距離センサの異常原因としてRaspberryPi3 Model BのCPU負荷が大きくなりすぎてしまったことや,振動等により配線の接触不良が発生したこと等が考えられる.

\subsection{コントロールラインを検出し損ねた原因}
我々のロボカーは,フォトリフレクタを用いてコントロールラインを検出する際に,ロバスト性向上のためにフォトリフレクタから100回データを取得し,その平均がしきい値を超えたかどうかでコントロールラインの白色部をを通過したかどうかを検出していた.そのため,コントロールラインを通過する速度,位置,タイミングによっては,100回計測している間に白色部を通過してしまい,コントロールラインの検出が正しく行えなかった可能性があると考えられる.

\subsection{ステアリング制御が振動を起こした原因}
我々が使用した距離センサはセンサの仕様上,距離の測定に最速でも$20\unit{ms}$かかる.今回,ロボカーにはこの距離センサを3つ搭載し,それぞれのセンサで距離を測定した後にその距離データに応じて機体を制御していた.すなわち,制御周期は最速でも$60\unit{ms}$はかかることになる.さらに,実際には他の処理による遅延も含まれるため,より制御周期は遅くなっていたと考えられる.これにより,タイヤの目標角速度を高く設定した時に制御周期が追いつかなくなってしまい,振動を起こしてしまったと考えられる.