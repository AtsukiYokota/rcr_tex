\section{感想と次年度への提言}
今回,私は主にロボカーのソフトウェア開発とチーム全体の統括を担当した.そこで最も痛感した点はプロジェクトマネジメントの大変さである.RCRはチームで進めるプロジェクトであり,円滑なプロジェクト進行には,各メンバ間での進捗状況や意見等の共有が不可欠である.しかし今回は,あるメンバの仕事が終わらないと他のメンバの仕事に着手できないという状況を生んでしまったり,一部のメンバに仕事が集中してしまい,他のメンバにその進捗状況がうまく伝わらなかった等といった様々な問題が生じてしまった.

主な原因として,例年に比べ10人と人数が多くチームが大きくなりすぎた点,時系列を意識した役割ないし仕事の分担が出来ていなかった点,大学院一般入試の有無や経験の差などによって全メンバに平等に仕事を割り振ることが難しくなった点,そしてミーティングでその差を十分に埋めきれず,各メンバ間で理解度に差が生じてしまった点等が挙げられる.

また今年度は,機体のベースが配布されたことにより例年よりも発表内容が少なくなったことに加え,人数が多かったため,発表の際に一人あたりの分量が少なくなってしまい,かつ内容が細かく分割されるため,聴衆に伝わりやすいスライド構成にすることが難しかった.プレゼンテーションの評価のために全員発表にしなければならないと言った事情はもちろん承知しているが,来年以降はこの点に多少なりとも何かしらの配慮があれば良いと考える.