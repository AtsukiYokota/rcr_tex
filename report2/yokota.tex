\documentclass[10pt,a4j]{jarticle}

\input{include/macro.tex}
\input{include/preamble.tex}

\begin{document}

\section{ROS}
\subsection{ROSとは}
ROS(Robot Operating System)とはOpen Source Robotics Foundationによって管理されているソフトウェア開発者のロボット・アプリケーション作成を支援するフレームワークである.具体的には,ハードウェア抽象化,デバイスドライバ,ライブラリ,視覚化ツール,メッセージ通信,パッケージ管理などが提供されている.つまりROSは汎用コンピュータ向けのOSではなく,汎用コンピュータ向けOS上で動作するメタOSとして捉えることができる.

ROSではプロセス(実行プログラム)はノードという単位で扱い,ノード間の通信はトピックと呼ばれる``Publisher/Subscriber''モデルで実現される.これにより,プログラミング言語や通信相手さえ意識することなく簡単にプロセス間通信を実現できる.また,これは各ノード間のインタフェース,即ちトピックの名前と型さえ決定すれば独立して開発を行うことができるという利点でもある.

以上より,本研究室ではRaspberryPi上にROSをインストール


\end{document}
