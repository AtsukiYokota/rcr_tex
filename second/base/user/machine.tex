\section{機体}
\refig{first}に配布されたロボカーのベース部を示す.本体の前方には前輪を駆動させ機体を旋回させるためのサーボモータを設置している.そして,本体の後方には後輪を駆動させ機体を動かすためのDCモータを設置し,その上に角速度を計測するためのロータリエンコーダを設置した.

ベース部の上部に他の部品を設置するために,加工と絶縁が容易な点を鑑み,ユニバーサルプレートを二層設置した.\refig{second},\refig{third}に搭載する部品のレイアウトを示す.\refig{second}より中央にはDCモータを制御するためのモータドライバ,後方には各部品に電力を供給するためのバッテリ,前方にはRaspberry Pi3 model Bに電力を供給するためのモバイルバッテリを設置した.そして,前方と左右にコースの壁の距離を計測するためのToFセンサを設置した.\refig{third}より,一番上にはRaspberry Pi3 model BとDC-DCコンバータを含めた電気回路を設置している.

最後に,\refig{forth}にロボカーの背面の写真を示す.\refig{forth}より,現在はまだ設置していないがロボカーの背面にゴールラインを読み取るためのフォトリフレクタを設置する予定である.

\begin{figure}[htb]
\centering
\includegraphics[width=0.5\hsize]{picture/eps/one.eps}
\caption{ロボカー本体}
\label{fig::first}
\end{figure}

\begin{figure}[htb]
\centering
\includegraphics[width=0.5\hsize]{picture/eps/two.eps}
\caption{ロボカー1段目}
\label{fig::second}
\end{figure}

\begin{figure}[htb]
\centering
\includegraphics[width=0.5\hsize]{picture/eps/three.eps}
\caption{ロボカー2段目}
\label{fig::third}
\end{figure}

\begin{figure}[htb]
\centering
\includegraphics[width=0.5\hsize]{picture/eps/four.eps}
\caption{ロボカー背面}
\label{fig::forth}
\end{figure}